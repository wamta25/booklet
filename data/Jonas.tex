Dynamic resource management allows programs running on supercomputers to adjust their number of compute node allocations at runtime. This dynamism offers potential improvements in both individual program efficiency and overall system utilization.

Despite growing interest in recent years, the adoption of dynamic resource management remains limited due to inadequate support from widely used resource managers, such as Slurm, and programming environments, such as MPI. Moreover, the development of flexible programs introduces significant increased programming complexity compared to static programs.

While recent research improved MPI's resource flexibility, significant challenges in programmability remain. In addition, MPI-based solutions typically focus on iterative parallelization, which limits their effectiveness for irregular and dynamic workloads.

Asynchronous Many-Task (AMT) programming is a promising alternative to MPI. By splitting computations into tasks that are dynamically scheduled by the runtime system, AMT is well suited to handle irregular and dynamic workloads. AMT's transparent resource management is ideal for dynamic resources, allowing the runtime system to redistribute tasks to accommodate node changes without additional programmer effort.

In this work, we compare the MPI proposal "Dynamic Processes with PSets (DPP)" and the APGAS+GLB AMT runtime system. We implement benchmarks in both environments to evaluate programmability and performance. We conduct experiments on 16 nodes to evaluate static and dynamic programs coupled with custom prototype resource managers. Results show that APGAS+GLB facilitates programmer productivity by transparently providing both load balancing and resource flexibility. In contrast, DPP achieves superior performance in handling node changes, but at the cost of increased programming complexity.
