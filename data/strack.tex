Parallel algorithms relying on synchronous parallelization libraries often experience adverse performance due to global synchronization barriers. Asynchronous many-task runtimes offer task futurization capabilities that minimize or remove the need for global synchronization barriers. Task futurization can improve overall algorithmic performance. However, some applications are better suited than others for migration to an asynchronous many-task model. This paper conducts a case study of the multidimensional Fast Fourier Transform to identify which applications will benefit from the asynchronous many-task model. Our principle focus is the popular FFTW library. We use the asynchronous many-task model HPX and a one-dimensional FFTW backend to implement multiple versions using different HPX features and highlight overheads and pitfalls during migration. Furthermore, a new HPX backend has been implemented and added to FFTW. The case study analyzes shared-memory scaling properties between our HPX-based parallelization and FFTW with its pthreads, OpenMP, and HPX backends. We examine the performance tradeoff of FFTW's algorithmic planning. The case study also compares FFTW's backends paired with MPI to a purely HPX-based implementation in a distributed environment.
