The Taskflow project addresses the long-standing question: "How can we make it easier for C++ developers to write efficient parallel programs with high productivity?" Modern computing applications rely on many parallel computing resources to achieve new performance milestones that were previously out of reach. However, programming these parallel computing resources is not an easy task because there are many technical details, such as abstraction, scheduling, concurrency, and load balancing. Many of these technical details are known difficult to program correctly.

To overcome this challenge, Taskflow develops a simple and powerful task graph programming model to enable efficient implementations of parallel decomposition strategies. Our programming model empowers users with both static and dynamic task graph constructions to implement various computational patterns, including in-graph control flow, composition, and on-the-fly tasking. Taskflow also introduces an efficient work-stealing scheduling algorithm that can efficiently balance the number of available workers with dynamically generated task parallelism. We have applied Taskflow to many industrial applications and achieved significant performance improvement through task graph parallelism. Taskflow is being used by many academic and industrial parallel computing applications, such as AMD Xilinx, Nvidia GameWorks, Tesseract Robotics, OSSIA sequencer, and so on.

Structure of the Talk:
The takeaway of the talk consists of the following five items:
\begin{enumerate}
\item Express your parallelism in the right way
\item Program task graph parallelism using Taskflow
\item Program dynamic task graph parallelism using Taskflow
\item Overcome the scheduling challenges
\item Demonstrate the efficiency of Taskflow
\end{enumerate}

We will start by explaining the importance to express the parallelism in the right way. Then, we will introduce two essential programming paradigms of Taskflow, static task graph parallelism and dynamic task graph parallelism. Next, we will discuss how Taskflow overcomes critical scheduling challenges in running static and dynamic task graphs. Finally, we will present successful industrial use cases of Taskflow.

