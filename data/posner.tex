In the rapidly evolving field of High-Performance Computing (HPC), the need for resource adaptivity is paramount, particularly in addressing the dynamic nature of irregular computational workloads. A key area of adaptivity lies within programming models, which typically offer limited support.

Fully adaptive programs are both malleable–capable of dynamically adjusting resources in response to external job scheduler requests–and evolving–autonomously deciding when and how to adjust resources, e.g., through automated decision-making. Previous adaptivity approaches typically relied on iterative workloads and required complex code modifications.

Asynchronous Many-Task (AMT) programming is emerging as a powerful alternative. In AMT, computations are split into fine-grained tasks, allowing transparent task relocation by the runtime system and unlocking significant potential for efficient adaptivity.

This work-in-progress proposes an extension to the existing AMT APGAS that recently incorporated malleability. Our extension adds evolving capabilities providing automatic and transparent resource adjustments to meet changing computational workloads at runtime. While our easy-to-use abstractions require only minimal code additions, adjustments such as process initialization and termination are managed automatically. Our extension is validated via a load-balancing library for irregular workloads.

We propose two heuristics for automatic computational load detection: one that uses CPU loads provided by the OS, and another that exploits detailed insights into task loads. We evaluate our approach using a novel synthetic benchmark that starts with a single task evolving into two irregular trees connected by a long sequential branch. Preliminary results are promising, indicating that the task-load-based heuristic outperforms the CPU-based one in responsiveness and effectiveness.
