We explore the data sparsity characteristics of various commonly used Radial Basis Function (RBF) kernels in the context of 3D unstructured mesh deformation. While RBF interpolation is a powerful method for generating high-quality adaptive meshes, solving the resulting boundary problems leads to large, dense linear systems that are computationally expensive and memory-intensive due to their cubic complexity. To address these challenges, we exploit the rank structure of the matrix operators by employing a Tile Low-Rank Cholesky-based solver, which approximates off-diagonal matrix tiles up to an application-specific accuracy threshold. Our study compares global support RBFs and compact support RBFs, focusing on their effects on rank distribution and numerical accuracy. Using realistic 3D geometries of SARS-CoV-2 viruses from the Protein Data Bank, we evaluate various RBF kernels, analyze the corresponding matrix rank structures, and assess the backward error resulting from low-rank approximations for different kernel types. We conduct experiments on various shared and distributed systems, demonstrating the performance scalability on massively parallel architectures. Leveraging the Hierarchical Computations on Manycore Architectures (HiCMA) library and the PaRSEC runtime system, we show how data sparsity accelerates large-scale mesh adaptation, providing valuable insights into the balance between computational efficiency and numerical accuracy.
