\begin{center}
\textit{Co-Authors: Gregor	Daiß}
\end{center} 
Modern hardware architectures are increasingly parallel, through both massively multicore CPUs and accelerators which often dominate the available compute on the newest nodes. Utilizing this hardware fully is increasingly difficult with traditional programming models. At the same time the hardware landscape has further diversified which increases the burden on application developers to run their applications anywhere. HPX, a tasking runtime with a focus on distributed applications and standards conformance, is ideally suited to tackle the first challenge. Kokkos, a performance portability layer, handles the second. However, using them together has not been straightforward until the integrations presented here. This talk presents the integration of HPX and Kokkos on multiple different levels with: 1. an HPX backend for Kokkos, which is the first asynchronous CPU backend for Kokkos; 2. HPX-Kokkos, a thin interoperability layer which combines primitives provided by HPX and Kokkos; and 3. The integration of all of the former into Octo-Tiger, an astrophysics application. The integration allows applications to make full use of the portability provided by Kokkos and move past the limitations of fork-join parallelism with the tasking provided by HPX. Finally, this talk will present the latest integration of the C++ std::execution proposal for asynchrony into HPX and Kokkos, and how it provides a path towards a standardized solution for portability and asynchrony in applications like Octo-Tiger.