MADNESS (Multiresolution ADaptive Numerical Environment for Scientific Simulation) started as an environment for fast and accurate numerical simulation in chemistry, but rapidly expanded to include applications in nuclear physics (HF and DFT for nuclei), boundary value problems, solid state physics, and atomic and molecular physics. It is portable from laptops to the largest supercomputers, and is open-source under GPL2 with developers/users in the US, Europe, Japan, and China. MADNESS provides a very high level of composition for science applications in terms of functions and operators rather than coefficients and matrix elements.

MADNESS employs adaptive multiresolution algorithms for fast computation with guaranteed precision, and separated representations for efficient computation in many dimensions.  To guarantee precision, every function has an independent and dynamically refined "mesh" and composing functions or applying operators can change the mesh refinement. These meshes are represented as 2$^{d}$-trees, where $d$ is the dimensionality of the problem. These trees are typically poorly balanced, hence computing with such trees on a parallel machine poses significant challenges that are addressed by the task-based MADNESS parallel runtime.

The MADNESS runtime has evolved into a powerful environment for the composition of a wide range of parallel algorithms, not just on trees, but on any distributed data structures, including the block-sparse tensors in TiledArray. The central elements of the parallel runtime are a) futures for hiding latency and managing dependencies, b) global namespaces with one-sided access so that applications are composed using names or concepts central to the application, c) non-process centric computing through remote method invocation in objects in global namespaces, and d) dynamic load balancing and data redistribution. An application in the MADNESS runtime can be viewed as a dynamically constructed DAG, with futures as edges.

After over 15 years of experience, despite many successes task-based composition on the MADNESS runtime has proven to be challenging to use robustly and has not fully migrated to hybrid architectures. This talk will examine both the successes and failures of the runtime and programming model, and will touch upon central elements in the design of the Template Task Graph (TTG, developed in collaboration with ICL team and Virginia Tech) that is the foundation of the next generation of MADNESS.

