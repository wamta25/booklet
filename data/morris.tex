\begin{center}
\textit{Co-Authors: Steven R.	Brandt}
\end{center} 
Guardian is a parallel programming language that compiles to Java, designed to work with the Javalin parallel programming library. Javalin provides a framework for parallel programming which is non-blocking, composable, orderable, compatible with fork-join pools, and free of deadlocks, all with minimal overhead. However, developing in Javalin requires an asynchronous continuation style of programming. This paradigm is indirect, involves deep nesting of lambdas, and requires adherence to contracts that are impossible to enforce in plain Java. Writing Javalin code in plain Java can be unintuitive and error-prone. Guardian aims to mend these concerns; the language introduces new, yet familiar syntax on top of Java to allow programming in a style more familiar to Java developers. It also prevents errors and pitfalls by enforcing Javalin's contracts at compile-time, guaranteeing safety.