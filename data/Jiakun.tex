MPI has been around for 30 years. Mainstream MPI applications feature coarse-grained, collective-style communication, no or limited overlap within a single global synchronization step, and single-thread communication. They have driven the MPI community’s optimization direction. Unfortunately, AMT systems communicate in a different manner: multiple threads can launch communication; messages are mostly point-to-point and fine-grained; there can be a large number of pending communication requests at the same time and no global synchronization points.

The Lightweight Communication Interface (LCI) is an experimental communication library aiming to push for efficient multithreaded irregular communication support. It has been integrated into two AMTs (HPX and PaRSEC) and has shown significant performance improvement in microbenchmarks and real-world applications at large scales. In this talk, we will present the newly developed LCI version 2 interface based on C++. It has the following main features (a) a simple but expressive interface to unify common communication primitives and completion mechanisms (b) a three-level resource hierarchy for library interoperating, multithreaded performance isolation, and large message multi-channel aggregation (c) explicit progress control and optional memory registration interface (d) a set of compilation and runtime variables to customize runtime behavior. We will also talk about how this interface maps to low-level network functionality and runtime behavior and how it fits into high-level AMT communication abstraction.
